% Options for packages loaded elsewhere
\PassOptionsToPackage{unicode}{hyperref}
\PassOptionsToPackage{hyphens}{url}
\documentclass[
]{article}
\usepackage{xcolor}
\usepackage[margin=1in]{geometry}
\usepackage{amsmath,amssymb}
\setcounter{secnumdepth}{5}
\usepackage{iftex}
\ifPDFTeX
  \usepackage[T1]{fontenc}
  \usepackage[utf8]{inputenc}
  \usepackage{textcomp} % provide euro and other symbols
\else % if luatex or xetex
  \usepackage{unicode-math} % this also loads fontspec
  \defaultfontfeatures{Scale=MatchLowercase}
  \defaultfontfeatures[\rmfamily]{Ligatures=TeX,Scale=1}
\fi
\usepackage{lmodern}
\ifPDFTeX\else
  % xetex/luatex font selection
\fi
% Use upquote if available, for straight quotes in verbatim environments
\IfFileExists{upquote.sty}{\usepackage{upquote}}{}
\IfFileExists{microtype.sty}{% use microtype if available
  \usepackage[]{microtype}
  \UseMicrotypeSet[protrusion]{basicmath} % disable protrusion for tt fonts
}{}
\makeatletter
\@ifundefined{KOMAClassName}{% if non-KOMA class
  \IfFileExists{parskip.sty}{%
    \usepackage{parskip}
  }{% else
    \setlength{\parindent}{0pt}
    \setlength{\parskip}{6pt plus 2pt minus 1pt}}
}{% if KOMA class
  \KOMAoptions{parskip=half}}
\makeatother
\usepackage{color}
\usepackage{fancyvrb}
\newcommand{\VerbBar}{|}
\newcommand{\VERB}{\Verb[commandchars=\\\{\}]}
\DefineVerbatimEnvironment{Highlighting}{Verbatim}{commandchars=\\\{\}}
% Add ',fontsize=\small' for more characters per line
\usepackage{framed}
\definecolor{shadecolor}{RGB}{248,248,248}
\newenvironment{Shaded}{\begin{snugshade}}{\end{snugshade}}
\newcommand{\AlertTok}[1]{\textcolor[rgb]{0.94,0.16,0.16}{#1}}
\newcommand{\AnnotationTok}[1]{\textcolor[rgb]{0.56,0.35,0.01}{\textbf{\textit{#1}}}}
\newcommand{\AttributeTok}[1]{\textcolor[rgb]{0.13,0.29,0.53}{#1}}
\newcommand{\BaseNTok}[1]{\textcolor[rgb]{0.00,0.00,0.81}{#1}}
\newcommand{\BuiltInTok}[1]{#1}
\newcommand{\CharTok}[1]{\textcolor[rgb]{0.31,0.60,0.02}{#1}}
\newcommand{\CommentTok}[1]{\textcolor[rgb]{0.56,0.35,0.01}{\textit{#1}}}
\newcommand{\CommentVarTok}[1]{\textcolor[rgb]{0.56,0.35,0.01}{\textbf{\textit{#1}}}}
\newcommand{\ConstantTok}[1]{\textcolor[rgb]{0.56,0.35,0.01}{#1}}
\newcommand{\ControlFlowTok}[1]{\textcolor[rgb]{0.13,0.29,0.53}{\textbf{#1}}}
\newcommand{\DataTypeTok}[1]{\textcolor[rgb]{0.13,0.29,0.53}{#1}}
\newcommand{\DecValTok}[1]{\textcolor[rgb]{0.00,0.00,0.81}{#1}}
\newcommand{\DocumentationTok}[1]{\textcolor[rgb]{0.56,0.35,0.01}{\textbf{\textit{#1}}}}
\newcommand{\ErrorTok}[1]{\textcolor[rgb]{0.64,0.00,0.00}{\textbf{#1}}}
\newcommand{\ExtensionTok}[1]{#1}
\newcommand{\FloatTok}[1]{\textcolor[rgb]{0.00,0.00,0.81}{#1}}
\newcommand{\FunctionTok}[1]{\textcolor[rgb]{0.13,0.29,0.53}{\textbf{#1}}}
\newcommand{\ImportTok}[1]{#1}
\newcommand{\InformationTok}[1]{\textcolor[rgb]{0.56,0.35,0.01}{\textbf{\textit{#1}}}}
\newcommand{\KeywordTok}[1]{\textcolor[rgb]{0.13,0.29,0.53}{\textbf{#1}}}
\newcommand{\NormalTok}[1]{#1}
\newcommand{\OperatorTok}[1]{\textcolor[rgb]{0.81,0.36,0.00}{\textbf{#1}}}
\newcommand{\OtherTok}[1]{\textcolor[rgb]{0.56,0.35,0.01}{#1}}
\newcommand{\PreprocessorTok}[1]{\textcolor[rgb]{0.56,0.35,0.01}{\textit{#1}}}
\newcommand{\RegionMarkerTok}[1]{#1}
\newcommand{\SpecialCharTok}[1]{\textcolor[rgb]{0.81,0.36,0.00}{\textbf{#1}}}
\newcommand{\SpecialStringTok}[1]{\textcolor[rgb]{0.31,0.60,0.02}{#1}}
\newcommand{\StringTok}[1]{\textcolor[rgb]{0.31,0.60,0.02}{#1}}
\newcommand{\VariableTok}[1]{\textcolor[rgb]{0.00,0.00,0.00}{#1}}
\newcommand{\VerbatimStringTok}[1]{\textcolor[rgb]{0.31,0.60,0.02}{#1}}
\newcommand{\WarningTok}[1]{\textcolor[rgb]{0.56,0.35,0.01}{\textbf{\textit{#1}}}}
\usepackage{graphicx}
\makeatletter
\newsavebox\pandoc@box
\newcommand*\pandocbounded[1]{% scales image to fit in text height/width
  \sbox\pandoc@box{#1}%
  \Gscale@div\@tempa{\textheight}{\dimexpr\ht\pandoc@box+\dp\pandoc@box\relax}%
  \Gscale@div\@tempb{\linewidth}{\wd\pandoc@box}%
  \ifdim\@tempb\p@<\@tempa\p@\let\@tempa\@tempb\fi% select the smaller of both
  \ifdim\@tempa\p@<\p@\scalebox{\@tempa}{\usebox\pandoc@box}%
  \else\usebox{\pandoc@box}%
  \fi%
}
% Set default figure placement to htbp
\def\fps@figure{htbp}
\makeatother
\setlength{\emergencystretch}{3em} % prevent overfull lines
\providecommand{\tightlist}{%
  \setlength{\itemsep}{0pt}\setlength{\parskip}{0pt}}
\usepackage{bookmark}
\IfFileExists{xurl.sty}{\usepackage{xurl}}{} % add URL line breaks if available
\urlstyle{same}
\hypersetup{
  pdftitle={PY1617: Foundations of R for Statistics and Data Science},
  pdfauthor={Dr Abigail Page \& Dr James Winters},
  hidelinks,
  pdfcreator={LaTeX via pandoc}}

\title{PY1617: Foundations of R for Statistics and Data Science}
\usepackage{etoolbox}
\makeatletter
\providecommand{\subtitle}[1]{% add subtitle to \maketitle
  \apptocmd{\@title}{\par {\large #1 \par}}{}{}
}
\makeatother
\subtitle{Component Notebook}
\author{Dr Abigail Page \& Dr James Winters}
\date{}

\begin{document}
\maketitle

{
\setcounter{tocdepth}{3}
\tableofcontents
}
\section{Introduction to R: Week 1}\label{introduction-to-r-week-1}

\begin{quote}
Welcome to the R component of \textbf{PY1617 Employability: Foundations
of R for Statistics and Data Science}
\end{quote}

This notebook brings together all material for the R component of the
module, spanning ten weeks of teaching.

In this first week, the focus is on \textbf{orientation}: understanding
what R is, why we are using it, how it fits into data science and
statistics and getting everything set up correctly on your computer.

\begin{center}\rule{0.5\linewidth}{0.5pt}\end{center}

\subsection{Overview of the R
Component}\label{overview-of-the-r-component}

This component runs across \textbf{ten teaching weeks}, starting from
the very basics of using R and progressing to writing functions and
simulating data by the end of the term.

Teaching is shared across the module:

\begin{enumerate}
\def\labelenumi{\arabic{enumi})}
\tightlist
\item
  \textbf{Weeks 1--5} are taught by Dr Abigail Page and focus on
  foundational R skills, data handling, and visualisation
\item
  \textbf{Weeks 7--10} are taught by Dr James Winters and focus on
  programming concepts and simulation
\end{enumerate}

By the end of the component, you will be able to use R confidently as a
tool for working with data in academic, research and workplace contexts.

\begin{center}\rule{0.5\linewidth}{0.5pt}\end{center}

\subsection{Aims of the Component}\label{aims-of-the-component}

The R component aims to develop \textbf{foundational skills in R} that
underpin data analysis, statistics and data science.

Specifically, the component aims to: 1) Build confidence in using R and
RStudio 2) Develop good habits for writing clear, reproducible code 3)
Introduce core data workflows used in statistics and research 4) Prepare
you for later statistics modules in Years 1 and 2 5) Develop digital and
data literacy skills valued in the workplace

\begin{center}\rule{0.5\linewidth}{0.5pt}\end{center}

\subsection{Learning Objectives}\label{learning-objectives}

By the end of this component, you should be able to:

\begin{enumerate}
\def\labelenumi{\arabic{enumi})}
\tightlist
\item
  Use \textbf{R and RStudio} confidently, including scripts, projects
  and packages\\
\item
  Understand and work with \textbf{data types, objects and data frames}
  in R\\
\item
  Apply a \textbf{tidy data workflow} to clean, transform and reshape
  data\\
\item
  Create and label \textbf{new and recoded variables}\\
\item
  \textbf{Summarise and visualise data} effectively using ggplot2\\
\item
  Use basic \textbf{programming concepts} (loops and functions) to
  automate tasks\\
\item
  Write \textbf{clear, reproducible code} suitable for academic and
  workplace contexts\\
\item
  Use these skills to \textbf{simulate data}
\end{enumerate}

\begin{center}\rule{0.5\linewidth}{0.5pt}\end{center}

\subsection{What Is Data Science?}\label{what-is-data-science}

Data science is the process of turning data into understanding and
insight.

This typically includes:

\begin{enumerate}
\def\labelenumi{\arabic{enumi})}
\tightlist
\item
  Managing data\\
\item
  Cleaning and preparing data\\
\item
  Exploring and visualising data\\
\item
  Modelling and interpreting results\\
\item
  Communicating findings clearly
\end{enumerate}

Data science is best thought of as a \textbf{workflow}, rather than a
single technique.

\subsubsection{The data science
workflow}\label{the-data-science-workflow}

Throughout this module, we will refer to the \textbf{data science cycle
(Figure 1.1)}, which describes how data are:

\begin{enumerate}
\def\labelenumi{\arabic{enumi})}
\tightlist
\item
  Imported: bringing data into R\\
\item
  Prepared and transformed: cleaning the data so you can do what you
  want with it.\\
\item
  Explored and visualised: explore patterns and trends\\
\item
  Modelled: apply statistical models\\
\item
  Communicated: report your results clearly in plots and tables
\end{enumerate}

\begin{figure}
\centering
\pandocbounded{\includegraphics[keepaspectratio]{workflow.png}}
\caption{Figure 1.1 Workflow Diagram from R4DS textbook}
\end{figure}

We can think of this as a cycle because you don't do the steps only
once, but you loop back and forth.

The first step in any data analysis is importing your data into R. This
usually involves loading data from a file into a data frame. Without
access to your data within R, you cannot perform any analysis. Once your
data is imported, it's important to tidy it. Tidy data has a consistent
structure where each column represents a variable and each row
represents an observation and each cell represents a value (Figure 1.2).
This organization makes it easier to work with the data, allowing you to
focus on analysis rather than data cleanup.

\begin{figure}
\centering
\pandocbounded{\includegraphics[keepaspectratio]{tidy-1.png}}
\caption{Figure 1.2 Variables, observations and values from R4DS
textbook}
\end{figure}

After tidying, the next step is often transforming your data.
Transformations can include filtering for specific observations (e.g.,
data from a particular participant or year), creating new variables from
existing ones (e.g., calculating speed from distance and time), or
computing summary statistics (e.g., counts, averages). The process of
tidying and transforming data is commonly called data wrangling, because
getting data into a usable form can be challenging.

With clean and well-structured data, you can begin generating insights
through visualisation and modelling. These approaches complement each
other:

\begin{itemize}
\item
  Visualisation is a human-centred activity. Well-designed
  visualisations can reveal unexpected patterns, suggest new questions
  or indicate that you may need additional data. While visualisations
  are invaluable for understanding, they rely on human interpretation
  and don't scale easily to large datasets.
\item
  Modelling provides a computational approach. Once your questions are
  clearly defined, models can answer them efficiently and at scale.
  However, models operate under assumptions and cannot challenge those
  assumptions on their own. They are powerful, but inherently limited in
  their ability to surprise.
\end{itemize}

Finally, communication is essential. The insights you derive from
analysis - whether through visualisations, models or summaries --- are
only useful if you can clearly explain them to others.

Throughout the process, programming supports every step. While you don't
need to be a programming expert, stronger programming skills allow you
to automate tasks, work more efficiently and tackle new problems more
effectively. These tools cover most of what you'll need in a data
science project/

This R component focuses primarily on the \textbf{foundational stages of
this cycle: importing, tidying and visualising data}. These are the
skills that everything else depends on. Data modelling and communication
will be focused on in other components and modules.

\begin{center}\rule{0.5\linewidth}{0.5pt}\end{center}

\subsection{What Is R?}\label{what-is-r}

\textbf{R} is a programming language designed specifically for
\textbf{data analysis, statistics and graphics}.\\
It was first developed in 1996 and is now widely used in academia,
research and industry.

R is: - A \textbf{command-driven} language - Highly flexible and
extensible - Widely used for statistical analysis and data visualisation

You may have encountered other statistical software such as SPSS, Stata,
or MATLAB. R differs from these in that it is: - Script-based rather
than menu-driven (you write things rather than click things) - Highly
customisable - Open source and free to use

\begin{center}\rule{0.5\linewidth}{0.5pt}\end{center}

\subsection{What Is RStudio?}\label{what-is-rstudio}

RStudio is the software environment we will use to work with R.

You can think of: - R as the engine which does the work\\
- RStudio as the dashboard which makes it easy to do the work

RStudio provides: - A script editor - A console for running code - Tools
for viewing data, plots and files - Integrated help and documentation

Once we have installed RStudio we no longer need to directly access R!

\subsubsection{Why we use R \& RStudio}\label{why-we-use-r-rstudio}

R \& RStudio is widely used because: - It is \textbf{free} and open
source - There is a large and active \textbf{R community} - Thousands of
\textbf{packages} extend R's functionality - Solutions to common
problems are easy to find online - It produces \textbf{high-quality,
publication-ready graphics}

\begin{center}\rule{0.5\linewidth}{0.5pt}\end{center}

\subsection{Challenges of Learning R}\label{challenges-of-learning-r}

Learning R can feel challenging at first, especially if you do not have
a programming background.

Common difficulties include: - A steep initial learning curve - Small
errors in code causing unexpected problems (e.g.~a missing closing
bracket or misplaced comma can be frustrating) - Not knowing how to
phrase questions when searching for help

This is normal.

\begin{quote}
\textbf{A useful rule of thumb:}\\
Knowing what to search for is a large part of working effectively in R.
AI tools make this a lot easier!
\end{quote}

Over time, you will build familiarity with common patterns and
solutions.

\begin{center}\rule{0.5\linewidth}{0.5pt}\end{center}

\subsection{Installing R and RStudio}\label{installing-r-and-rstudio}

Before you can start using R, you need to install \textbf{both R and
RStudio}.

\subsubsection{Installing R}\label{installing-r}

\begin{enumerate}
\def\labelenumi{\arabic{enumi}.}
\tightlist
\item
  Go to: \url{https://cran.r-project.org}\\
\item
  Select your operating system (Windows, macOS, or Linux)\\
\item
  Download and install the \textbf{latest version} of R
\end{enumerate}

\begin{quote}
Always install R \textbf{before} installing RStudio.
\end{quote}

\begin{center}\rule{0.5\linewidth}{0.5pt}\end{center}

\subsubsection{Installing RStudio}\label{installing-rstudio}

\begin{enumerate}
\def\labelenumi{\arabic{enumi}.}
\tightlist
\item
  Go to: \url{https://www.rstudio.com/products/rstudio/download/}\\
\item
  Download \textbf{RStudio Desktop (free version)}\\
\item
  Install the latest version for your operating system
\end{enumerate}

Once installed, you will \textbf{open RStudio}, not R itself.

Opening RStudio automatically starts R in the background.

Finally You will need to download Rtools from Cran:
\url{https://cran.r-project.org/bin/windows/Rtools/}. Make sure this is
installed before moving on.

\begin{center}\rule{0.5\linewidth}{0.5pt}\end{center}

\subsection{First Steps in RStudio}\label{first-steps-in-rstudio}

When you open RStudio for the first time, you will see several
\textbf{panes} (also called panels). Each has a different purpose to
help you work efficiently in R.

\begin{itemize}
\tightlist
\item
  The \textbf{console} (where code runs) on the left
\item
  The \textbf{environment} (where objects are stored) on the top right
\item
  Panels for plots, files, and help on the bottom right
\end{itemize}

\subsubsection{Console}\label{console}

The \textbf{console} is where you \textbf{type and run R commands
directly}. We will play around with this at the end of the session. The
console executes your commands immediately and shows the results.

\subsubsection{Envrionment and history}\label{envrionment-and-history}

The Environment tab shows all the objects you have loaded into R or
created (variables, data frames, lists, etc). The History tab keeps a
record of all the commands you have run in the console. You can click on
it and see what you have just done, and re-run it.

\subsubsection{Files, plots, packages, helps and
viewer}\label{files-plots-packages-helps-and-viewer}

This is were you can view lots of useful pieces of information and
access your files:

\begin{enumerate}
\def\labelenumi{\arabic{enumi})}
\tightlist
\item
  Files: Lets you browse files in your project folder. You can import
  data from here.
\item
  Plots: Displays graphs and visualisations you create. This pops up
  automatically when you make a plot.
\item
  Packages: Displays which R packages are installed and loaded. What you
  can use to install more packages (we will get to this)
\item
  Help: Shows documentation for functions you search.
\item
  Viewer: Used for HTML outputs and interactive visualisations.
\end{enumerate}

These tabs help you navigate your project, visualize results and find
help quickly. We will come to each of these sections in turn.

First, there is a final section of RStudio which is only viable when you
open up a R Script.

\subsubsection{R scripts}\label{r-scripts}

An R script is a file where you can write, save and edit multiple lines
of R code. Like a word document for R code! Scripts have the file
extension .R and are saved in a folder on your computer.

You can either open a blank script: - Go to File (top left piece of
paper with a green plus) - New File - R Script

This creates a script where you can write and save your code.

OR you can load a pre-made R script which you or someone else have made
to run. You can download the pre-made R Script for Week 1 session here:

\href{https://brightspace.brunel.ac.uk/d2l/le/lessons/67180/topics/2028425}{Download
the R Script}

\begin{center}\rule{0.5\linewidth}{0.5pt}\end{center}

\subsection{Basic R Expressions}\label{basic-r-expressions}

You can: - Type directly into the \textbf{console}, or\\
- Write code in a \textbf{script}, which can be \textbf{saved, shared
and reproduced}.

\begin{quote}
💡 Tip: If you write directly in the console, your work is \textbf{not
saved} when RStudio closes.
\end{quote}

So while we will show you how to write script directly into the console
today, in future we will be working with scripts.

\subsubsection{Writing in the console}\label{writing-in-the-console}

In R, \texttt{\textless{}-} is the \textbf{assignment operator} (not
\texttt{=}).\\
- Example: \texttt{x\ \textless{}-\ 500} tells R ``x equals 500.''\\
- Whenever you type \texttt{x}, R will read it as 500.

\textbf{Steps:} 1) In the console (after the \texttt{\textgreater{}}),
type:\\
2) ``x \textless- 500'' 3) press enter 4) write x in the next line 5)
press enter

it will look like this:

\begin{Shaded}
\begin{Highlighting}[]
\NormalTok{x }\OtherTok{\textless{}{-}} \DecValTok{500} 
\NormalTok{x}
\end{Highlighting}
\end{Shaded}

\begin{verbatim}
## [1] 500
\end{verbatim}

\begin{quote}
\textbf{Question}: when you select ``x'' what is the output?
\end{quote}

\begin{quote}
\textbf{Activity}: practice with different numbers and letters
\end{quote}

In the top right corner of the console you will see a \textbf{brush}
icon. If you click on it, it clears your console environment. Do this
now.

You now have a blank screen so have lost your work. If you go to the
history tab now you can see your previous code, and doubling clicking it
will bring it back to the Console for you to re-run!

But because you cannot save your work in the console we recommend you
always write code in an R script.

\subsubsection{Writing in a script}\label{writing-in-a-script}

Lines starting with \texttt{\#} are \textbf{comments}. These are notes
you leave for yourself or others to help understand what it is you have
done

\begin{itemize}
\tightlist
\item
  R \textbf{ignores them}, so they are not run.\\
\item
  They can say anything you like
\item
  They are essential - you will thank yourself later for a well
  commented script!
\end{itemize}

Below is what working in a script looks like.

\begin{Shaded}
\begin{Highlighting}[]
 \CommentTok{\#To run the code below just select the lines and press \textquotesingle{}run\textquotesingle{} above (or control/command + enter). }
\CommentTok{\# You can run line by line, or select it all at once.}

\NormalTok{x }\OtherTok{\textless{}{-}} \DecValTok{10} \CommentTok{\# telling R that x now equals 10  (this FYI is an example of a comment on code {-} the right of the \# is ignored)}
\FunctionTok{print}\NormalTok{ (x) }\CommentTok{\# printing this in the console}
\end{Highlighting}
\end{Shaded}

\begin{verbatim}
## [1] 10
\end{verbatim}

\begin{Shaded}
\begin{Highlighting}[]
\CommentTok{\# so this has done the same as above, but now your work is saved here. }

\NormalTok{y }\OtherTok{\textless{}{-}} \DecValTok{10} \SpecialCharTok{+} \DecValTok{20} \CommentTok{\# telling R that y now equals 10 + 20}
\FunctionTok{print}\NormalTok{ (y)  }\CommentTok{\# printing this in the console {-} you can run both these lines at once by selecting both. }
\end{Highlighting}
\end{Shaded}

\begin{verbatim}
## [1] 30
\end{verbatim}

\begin{Shaded}
\begin{Highlighting}[]
\NormalTok{z }\OtherTok{\textless{}{-}} \StringTok{"hello"} \CommentTok{\# telling R that z now equals "hello"}
\FunctionTok{print}\NormalTok{(z) }\CommentTok{\# printing this in the console}
\end{Highlighting}
\end{Shaded}

\begin{verbatim}
## [1] "hello"
\end{verbatim}

\begin{Shaded}
\begin{Highlighting}[]
\CommentTok{\# you can also get R to print by highlighting the text and press enter/run (called auto{-}printing) which is often easier to do!}

\NormalTok{a }\OtherTok{\textless{}{-}} \StringTok{"hello"}
\NormalTok{b}\OtherTok{\textless{}{-}} \StringTok{"world"}
\NormalTok{a }
\end{Highlighting}
\end{Shaded}

\begin{verbatim}
## [1] "hello"
\end{verbatim}

\begin{Shaded}
\begin{Highlighting}[]
\NormalTok{b   }\CommentTok{\# run all four lines at once }
\end{Highlighting}
\end{Shaded}

\begin{verbatim}
## [1] "world"
\end{verbatim}

\section{Workflow in R: Week 2}\label{workflow-in-r-week-2}

\subsection{Files, folders, and
projects}\label{files-folders-and-projects}

\subsection{Working directories}\label{working-directories}

\subsection{R scripts and packages}\label{r-scripts-and-packages}

\section{Data in R: Week 3}\label{data-in-r-week-3}

\subsection{Data types and objects}\label{data-types-and-objects}

\subsection{Vectors and data frames}\label{vectors-and-data-frames}

\subsection{Functions in R}\label{functions-in-r}

\section{Data Manipulation in R: Week
4}\label{data-manipulation-in-r-week-4}

\subsection{The tidyverse}\label{the-tidyverse}

\subsection{Recoding and transforming
variables}\label{recoding-and-transforming-variables}

\subsection{Missing data (NAs)}\label{missing-data-nas}

\subsection{Wide and long data}\label{wide-and-long-data}

\section{Data Visualisation in R: Week
5}\label{data-visualisation-in-r-week-5}

\subsection{Summarising data}\label{summarising-data}

\subsection{Introduction to ggplot2}\label{introduction-to-ggplot2}

\subsection{Interpreting plots}\label{interpreting-plots}

\section{Conditional Programming: Week
6}\label{conditional-programming-week-6}

\subsection{Logical values}\label{logical-values}

\subsection{if and else statements}\label{if-and-else-statements}

\section{Pipes and Loops: Week 7}\label{pipes-and-loops-week-7}

\subsection{Pipes (\%\textgreater\%)}\label{pipes}

\subsection{for loops}\label{for-loops}

\subsection{Repeating operations}\label{repeating-operations}

\section{Writing Functions from scratch: Week
8}\label{writing-functions-from-scratch-week-8}

\subsection{Why write functions?}\label{why-write-functions}

\subsection{Function structure}\label{function-structure}

\subsection{Writing your first
function}\label{writing-your-first-function}

\section{Simulating Data: Week 9}\label{simulating-data-week-9}

\subsection{Why simulate data?}\label{why-simulate-data}

\subsection{Generating random data}\label{generating-random-data}

\subsection{A simple simulation
example}\label{a-simple-simulation-example}

\end{document}
